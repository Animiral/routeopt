%%%%%%%%%%%%%%%%%%%%%%%%%%%%%%%%%%%%%%%%%
% Journal Article
% LaTeX Template
% Version 1.4 (15/5/16)
%
% This template has been downloaded from:
% http://www.LaTeXTemplates.com
%
% Original author:
% Frits Wenneker (http://www.howtotex.com) with extensive modifications by
% Vel (vel@LaTeXTemplates.com)
%
% License:
% CC BY-NC-SA 3.0 (http://creativecommons.org/licenses/by-nc-sa/3.0/)
%
%%%%%%%%%%%%%%%%%%%%%%%%%%%%%%%%%%%%%%%%%

%----------------------------------------------------------------------------------------
%	PACKAGES AND OTHER DOCUMENT CONFIGURATIONS
%----------------------------------------------------------------------------------------

\documentclass{article}

\usepackage[sc]{mathpazo} % Use the Palatino font
\usepackage[T1]{fontenc} % Use 8-bit encoding that has 256 glyphs
\linespread{1.05} % Line spacing - Palatino needs more space between lines
\usepackage{microtype} % Slightly tweak font spacing for aesthetics

\usepackage[english]{babel} % Language hyphenation and typographical rules

\usepackage[hmarginratio=1:1,top=32mm,columnsep=20pt]{geometry} % Document margins
\usepackage[hang, small,labelfont=bf,up,textfont=it,up]{caption} % Custom captions under/above floats in tables or figures
\usepackage{booktabs} % Horizontal rules in tables

\usepackage{lettrine} % The lettrine is the first enlarged letter at the beginning of the text

\usepackage{enumitem} % Customized lists
\setlist[itemize]{noitemsep} % Make itemize lists more compact

\usepackage{abstract} % Allows abstract customization
\renewcommand{\abstractnamefont}{\normalfont\bfseries} % Set the "Abstract" text to bold
\renewcommand{\abstracttextfont}{\normalfont\small\itshape} % Set the abstract itself to small italic text

\usepackage{titlesec} % Allows customization of titles
\renewcommand\thesection{\Roman{section}} % Roman numerals for the sections
\renewcommand\thesubsection{\roman{subsection}} % roman numerals for subsections
\titleformat{\section}[block]{\large\scshape\centering}{\thesection.}{1em}{} % Change the look of the section titles
\titleformat{\subsection}[block]{\large}{\thesubsection.}{1em}{} % Change the look of the section titles

\usepackage{fancyhdr} % Headers and footers
\pagestyle{fancy} % All pages have headers and footers
\fancyhead{} % Blank out the default header
\fancyfoot{} % Blank out the default footer
\fancyhead[C]{Running title $\bullet$ May 2016 $\bullet$ Vol. XXI, No. 1} % Custom header text
\fancyfoot[RO,LE]{\thepage} % Custom footer text

\usepackage{titling} % Customizing the title section

\usepackage{hyperref} % For hyperlinks in the PDF

%----------------------------------------------------------------------------------------
%	TITLE SECTION
%----------------------------------------------------------------------------------------

\setlength{\droptitle}{-4\baselineskip} % Move the title up

\pretitle{\begin{center}\Huge\bfseries} % Article title formatting
\posttitle{\end{center}} % Article title closing formatting
\title{Analysis of Path Search with Restrictions} % Article title
\author{%
\textsc{Peter Neubauer} \\[1ex] % Your name
\normalsize Vienna University of Technology \\ % Your institution
\normalsize \href{mailto:peterneubauer2@gmail.com}{peterneubauer2@gmail.com} % Your email address
%\and % Uncomment if 2 authors are required, duplicate these 4 lines if more
%\textsc{Jane Smith}\thanks{Corresponding author} \\[1ex] % Second author's name
%\normalsize University of Utah \\ % Second author's institution
%\normalsize \href{mailto:jane@smith.com}{jane@smith.com} % Second author's email address
}
\date{\today} % Leave empty to omit a date
\renewcommand{\maketitlehookd}{%
\begin{abstract}
\noindent In the domain of commercial flight planning, we deal with the hard problem of route optimization. Airlines, who operate flights between cities all over the world, aim to save costs while abiding by all laws and regulations.
In this work, we implement a stripped-down and simplified variant of an industry-standard shortest-path algorithm. We examine the performance of the search with respect to airspace restrictions published by ATC (air traffic control) authorities.
\end{abstract}
}

%----------------------------------------------------------------------------------------

\begin{document}

% Print the title
\maketitle

%----------------------------------------------------------------------------------------
%	ARTICLE CONTENTS
%----------------------------------------------------------------------------------------

\section{Problem Definition}
The goal of route optimization is to find a flight route with minimal or close to minimal cost in the given scenario.

The flight route is flown along a series of waypoints, which are connected by airways called *segments*. These waypoints and segments form the nodes and edges in our problem *airspace graph*.

Not every path through the airspace is permitted by regulations. The flight path must adhere to a list of rules set out by traffic control authorities. These rules may theoretically impose any formulation of arbitrary conditions on the journey, imposing a lot of complexity.

The cost of the route is measured in fuel and time. A *cost index* is applied to put the two measures into relation, yielding a total measurable cost.

In the real world, the problem is additionally complicated by factors that we do not consider in this work:

\begin{itemize}

\item Fast-changing weather conditions that affect the cost and are predicted ahead of time

\item{Overflight fees for flying through a country's airspace}

\item Climbs and descents do not necessarily begin and end at waypoint coordinates in the airspace graph

\item The flight route must be covered by alternate airports in range in case of an emergency

\item Free-route airspaces allow arbitrary flight trajectories besides the defined airspace network

\item Aircraft performance depends on the properties of the aircraft, mainly its model supplied by the manufacturer

\end{itemize}

\subsection{The Airspace Graph}

\subsection{Airspace Sectors}

\subsection{Airspace Restrictions}

\subsection{Performance Cost}

\subsection{The Path Avoiding Forbidden Pairs Problem}

\section{Reference Solution}
We examine the performance of a simple solution using Dijkstra's Algorithm for finding a shortest path.
In addition, we implement restrictions with an iterative approach. Every time we encounter a shortest-path solution that is disqualified for violating a restriction, we adapt the airspace graph accordingly and retry.
This approach was inspired by an impementation used in the industry.

\subsection{Dijkstra's Algorithm}

(Illustration and pseudocode)

\subsection{Formalized Restrictions}

(Illustration and pseudocode)

%------------------------------------------------

\section{Performance Analysis}
In this work, we examine the impact of airspace restrictions on the run-time performance of the reference solution in a real-world context.

Possible items to explore:

\begin{itemize}
\item how the size of the airspace graph or constraints list affects the run time
\item different search algorithms (e.g. with and without heuristics)
\end{itemize}

%----------------------------------------------------------------------------------------
%	REFERENCE LIST
%----------------------------------------------------------------------------------------

\begin{thebibliography}{99} % Bibliography - this is intentionally simple in this template
 
\end{thebibliography}

%----------------------------------------------------------------------------------------

\end{document}
